\section{Qué é KiCad?}\label{quuxe9-uxe9-kicad}

Unha breve descripción da suite KiCad KiCad es una suite de diseño
electrónico automatizado \footnote{EDA suite en inglés}. KiCad permite o
diseño tanto de esquemas de circuitos como das placas de circuito
impreso a nivel profesional. Hai versións de KiCada dispoñibles para
Windows, Linux, Apple OS X. A suite está dispoñible para Windows, Linux
e Apple OS X. E un programa gratuito distribuido baixo licencia
\textbf{GNU GPL v2}.

Mellor ainda, a suite KiCad é a elexida polo CERN para o desenvolvemento
e diseño de electrónica. É de esperar que con este respaldo a suite
mellore ainda mais.

\section{Qué imos facer? Qué se describe en este
documento?}\label{quuxe9-imos-facer-quuxe9-se-describe-en-este-documento}

Imos deseñar un shield para Arduino.

Daremos unha visión xeral da funcionalidade da suite KiCad pero non
vamos a redactar un tutorial detallado nin de deseño de circuitos nin do
mesmo KiCad. Daremos as pistas xustiñas para empezar a traballar con
KiCad, se queredes un tutorial detallado, en youtube tedes un super
recomendable, elaborado por
\href{https://www.youtube.com/playlist?list=PL1Hs_F1k2mdRVYDtdWd7tQKDZTfxop7np}{TutoElectro}

Tamén comentaremos de pasada como fumos desenvolvendo iste proxecto.

\section{Instalación de KiCad (en
Ubuntu)}\label{instalaciuxf3n-de-kicad-en-ubuntu}

Para instalar o KiCad en Ubuntu basta con facer o típico ciclo de
instalación:

\begin{verbatim}
sudo apt-get install kicad
\end{verbatim}

Se queremos estar á última temos o ppa de Monsieur Reynaud dispoñible:

\begin{verbatim}
sudo apt-add-repository ppa:js-reynaud/ppa-KiCad
sudo apt-get update
sudo apt-get install kicad
\end{verbatim}

Nos escollimos esta opción.

Se non usades un linux baseado en Debian, teredes que consultar na rede
como facer a instalación para o voso sistema operativo. De todolos
xeitos a instalación é moi doada, donde podemos atopar algún problemiña
é na instalación das bibliotecas de compoñentes que vos contaremos cos
mais detalle mais adiante.

\section{Configuración de directorios para este
proxecto}\label{configuraciuxf3n-de-directorios-para-este-proxecto}

Además de desenvolver o proxecto con KiCad queremos ter o proxecto
dispoñible en github.

Agora que temos KiCad instalado imos preparar un directorio de traballo
ao que chamamos \textbf{sandboxShield}.

O directorio \textbf{sandboxShield} será o ``repositorio'' ou depósito
do noso proxecto para git. Contén os seguintes subdirectorios:

doc

: Contén a documentación do proxecto (o que estás a leer agora mesmo)
redactada en \href{http://pandoc.org/}{Pandoc}

kicad

: Contén o proxecto KiCad

Unha vez que temos preparado o directorio do proxecto activamos git para
iniciar o control de versións.

\begin{longtable}[c]{@{}l@{}}
\toprule
Describir a configuración de git??\tabularnewline
\bottomrule
\end{longtable}

\section{Biblioteca de compoñentes incluindo un shield para
Arduino}\label{biblioteca-de-compouxf1entes-incluindo-un-shield-para-arduino}

As bibliotecas de KiCad están organizadas en duas partes:

\begin{itemize}
\item
  Un fichero que contén os símbolos dos compoñentes para usarse no
  editor de esquemas electrónicos \textbf{Eescheme}
\item
  As pegadas dos compoñentes electrónicos, e decir a forma que ten que
  ter a pista da placa de circuito impreso (\emph{PCB}) para poder
  soldar o compoñente.
\end{itemize}

O KiCad non trae por defecto unha biblioteca de compoñentes que inclúa
shields de Arduino. Pero non hai problema hai bibliotecas que podemos
descarregar da rede.

Unha biblioteca moi completa é a de Freetronics que podemos atopar tamén
en github en:

\url{https://github.com/freetronics/freetronics_KiCad_library.git}

As bibliotecas de KiCad poden estar almacenadas en diferentes
directorios do noso ordenador. Poderíamos engadir as bibliotecas que
usemos en algún subdirectorio de \emph{/usr/share/kicad} ou de
\emph{/usr/local/share}. Esta podería ser unha boa estratexia en un
servidor compartido por varios usuarios. Tamén poderíamos descarregar
todas as bibliotecas a un directorio común do noso \emph{home}.

Pero como estamos facendo un control de versións do noso proxecto con
git a propia páxina da biblioteca suxírenos o xeito mais adoitado de
facer a instalación: coma un submódulo git do noso proxecto.

\begin{longtable}[c]{@{}l@{}}
\toprule
Describir as vantaxes de usar un git submodule\tabularnewline
\bottomrule
\end{longtable}

\subsection{Outra biblioteca moi
currada}\label{outra-biblioteca-moi-currada}

\url{git://smisioto.eu/KiCad_libs.git}

\subsection{Ainda mais bibliotecas}\label{ainda-mais-bibliotecas}

\url{http://www.kicadlib.org/}

\section{Instalación das
bibliotecas}\label{instalaciuxf3n-das-bibliotecas}

\url{http://www.arunet.co.uk/tkboyd/ele2pcbka.htm}

\section{Tutorial}\label{tutorial}

\begin{longtable}[c]{@{}l@{}}
\toprule
Pantalla xeral de KiCad opcións, citar a lista de hotkeys\tabularnewline
\bottomrule
\end{longtable}

Abrimos un novo proxecto: File::New Project (Ctrl+N)
\textbf{sandbox\_shield}

O primeiro que imos facer é o esquema do circuito. Para isto temos que
usar a ferramenta \emph{Eeschema} que podemos atopar en tres lugares
diferentes \footnote{Isto de ter varios xeitos de facer unha cousa é
  habitual en KiCad como iremos vendo} na barra de iconos de
ferramentas, no menú de KiCad no título da fiestra, ou có atallo
\textbf{Ctrl+E}.

Abrimos eescheme e creamos un novo fichero de esquema.

\section{Requisitos}\label{requisitos}

1 x LDR 3-4 x Botóns 1 x RGB 1 x LDR 1 x Pines servo 1 x sensor temp 2 x
potenciómetros 1 x LED bermello 1 x LED amarelo 1 x LED verde

\section{Meta}\label{meta}
