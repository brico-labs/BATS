%%
%% This is file `tikzposter-template.tex',
%% generated with the docstrip utility.
%%
%% The original source files were:
%%
%% tikzposter.dtx  (with options: `tikzposter-template.tex')
%%
%% This is a generated file.
%%
%% Copyright (C) 2014 by Pascal Richter, Elena Botoeva, Richard Barnard, and Dirk Surmann
%%
%% This file may be distributed and/or modified under the
%% conditions of the LaTeX Project Public License, either
%% version 2.0 of this license or (at your option) any later
%% version. The latest version of this license is in:
%%
%% http://www.latex-project.org/lppl.txt
%%
%% and version 2.0 or later is part of all distributions of
%% LaTeX version 2013/12/01 or later.
%%

\documentclass[20pt]{tikzposter} %Options for format can be included here

\usepackage{bookman}

% Esto es para poder escribir acentos directamente:
\usepackage[utf8]{inputenc}
% Esto es para que el LaTeX sepa que el texto está en español:
\usepackage[spanish]{babel}


 % Title, Author, Institute
\title{KiCAD Cheat Sheet}
\author{salvari}
\institute{BricoLabs}
\titlegraphic{LogoGraphic Inserted Here}


 %Choose Layout
\usetheme{Default}

\begin{document}

 % Title block with title, author, logo, etc.
\maketitle
\block{KiCAD}{

  KiCad é unha suite de deseño electrónico automatizado (EDA Suite en
  inglés). KiCad permite o deseño tanto de esquemas de circuitos como
  das placas de circuito impreso a nivel profesional. A suite está
  dispoñible para Windows, Linux e Apple OS X. É un programa gratuito
  e libre distribuido baixo licencia GNU GPL v2.

  Mellor ainda, a suite KiCad é a elexida polo CERN para o
  desenvolvemento e deseño de electrónica. É de esperar que con este
  respaldo a suite mellore ainda mais.
\\
\innerblock{Compoñentes da suite KiCAD}{
  \begin{description}
  \item[Eescheme] O editor dos esquemas eléctricos
  \item[Pcbnew] O editor das placas de circuito impreso (PCB en inglés)
  \item[Schematic Library Editor] O editor dos compoñentes esquemáticos
  \item[PCB footprint editor] O editor de ``pegadas'' dos compoñentes
  \item[GerbView] Un visor de ficheiros Gerb
  \item[Bitmap2Component] Conversor de bitmaps a compoñentes de
    Eescheme ou de Pcbnew
  \item[Pcb calculator] Unha calculadora de circuitos impresos con
    distintas utilidades
  \item[Pl editor] Un editor de plantillas para as follas de circuitos
    impresos ou de circuitos impresos
  \end{description}
}

}
 \begin{columns}

 % FIRST column
\column{0.5}% Width set relative to text width

\block{Eescheme}{
  \begin{tabular}{l|l|c}
    Axuda & Help & ?\\
    Aumentar zoom & Zoom In & F1\\
    Disminuir zoom & Zoom Out & F2\\
    Redibuxar zoom & Zoom Redraw & F3\\
    Centrar zoom & Zoom Center & F4\\
    Encaixar na pantalla & Fit on Screen & Home\\
    Borrar item & Delete Item & Del\\
    Repetir derradeiro item & Repeat Lest Item & Ins\\
    Mover ou arrastrar bloque & Move Block -> Drag Block & Tab\\
    Mover item & Move Item & M\\
    Copiar compoñente ou etiqueta & Copy Component or Label & C\\
    Arrastrar item & Drag Item & G\\
    Engadir compoñente & Add Component & A\\
    Engadir Alimentación & Add Power & P\\
    Xirar item & Rotate Item & R\\
    Invertir compoñente no eixo X & Mirror X Component & X\\
    Invertir compoñente no eixo Y & Mirror Y Component & Y\\
    Orientación por defecto de compoñente & Orient Normal Component & N\\
    Editar item & Edit Item & E\\
    Editar valor & Edit Value & V\\
    Editar referencia & Edit Reference & U\\
    Editar pegada & Edit Footprint & F\\
    Comenzar cableado & Begin Wire & W\\
    Comenzar Bus & Begin Bus & B\\
    Finalizar cableado ou bus & End Line Wire Bus & K\\
    Engadir etiqueta & Add Label & L\\
    Engadir etiqueta xerarquica & Add Hierarchical Label & H\\
    Engadir conexión de cables & Add Junction & J\\
    Engadir marca de no conexión & Add No Connect Flag & Q\\
    Engadir folla & Add Sheet & S\\
    Engadir conexión & Add Wire Entry & Z\\
    Engadir entrada de bus & Add Bus Entry & /\\
  \end{tabular}
}
% \note{Note with default behavior}
% \note[targetoffsetx=12cm, targetoffsety=-1cm, angle=20, rotate=25]
% {Note \\ offset and rotated}

 % First column - second block
\block{Fluxo de traballo}{
  \begin{list}{}{}
  \item 
  \end{list}
}

 % First column - third block
\block{Sample Block 4}{T\\E\\S\\T}

 % SECOND column
\column{0.5}
 %Second column with first block's top edge aligned with with previous column's top.

 % Second column - first block
\block{Pcbnew}{
  \begin{tabular}{l|l|c}
    Cambiar unidades & Switch Units & CTRL+U\\
    Resetear coordinadas locales & Reset Local Coordinates & Space\\
    Desfacer & Undo & Ctrl+Z\\
    Volver a facer & Redo & CTRL-Y\\
    Modo seguimento & Track Display Mode & K\\
    Borrar & Delete & Del\\
    Borrar segmento & Delete Track Segment & BkSp\\
    Engadir nova pista & Add New Track X\\
    Engadir via & Add Via & V\\
    Engadir microvia & Add Microvia & CTRL+V\\
    Cambiar posición da pista & Switch Track Posture & /\\
    Arrastrar pista manter pend & Drag Track, Keep Slope &  D\\
    Colocar compoñente & Place Item & P\\
    Copiar item & Copy Item & C\\
    Finalizar pista & End Track & End\\
    Move item & Move Item & M\\
    Invertir item & Flip Item & F\\
    Rotar item & Rotate Item & R\\
    Arrastrar item & Drag Item & G\\
    Buscar compoñente & Get and Move Footprint & T\\
    Bloquear pegada & Lock/Unlock Footprint & L\\
    Salvar tarxeta & Save Board & Ctrl+S\\
    Cargar tarxeta & Load Board & Ctrl+L\\
    Atopar item & Find Item & Ctrl+F\\
    Editar item & Edit Item & End\\
    Ir á capa de cobre & Switch to Cooper Layer & PgDn\\
    Ir á capa interna 1 & Switch to Inner Layer 1 & F5\\
    Ir á capa interna 2 & Switch to Inner Layer 2 & F6\\
    Ir á capa de compoñentes & Switch to Component Layer & PgUp\\
    Ir á siguiente capa & Switch to Next Layer & +\\
    Ir á capa previa & Switch to Previous Layer & -\\
    Engadir módulo & Add Module & O\\
    Camb. á anch. de pista sig. & Switch Track Width to Next & W\\
    Camb. á anch. de pista prev & Switch Track Width to Previous & Ctrl+W\\
    Modo de alto contraste & Switch Highcontrast Mode & H\\
    Invertir bloque (espello) & Invert (mirror) Block & Shift-LeftMouseB\\
  \end{tabular}
}

 % Second column - second block
\block[titlewidthscale=0.6, bodywidthscale=0.8]
{Variable width title}{Block with smaller width.}

 % Second column - third block
\block{}{Block with no title}

 % Second column - A collection of blocks in subcolumn environment.
\begin{subcolumns}
    \subcolumn{0.27} \block{1}{First block.} \block{2}{Second block}
    \subcolumn{0.4} \block{Sub-columns}{Sample subblocks\\Second subcolumn}
    \subcolumn{0.33} \block{4}{Fourth} \block{}{Final Subcolumn block}
\end{subcolumns}

 % Bottomblock
\block{Final Block in column}{
    Sample block.
}
\end{columns}
% \block[titleleft, titleoffsetx=2em, titleoffsety=1em, bodyoffsetx=2em,%
%  bodyoffsety=-2cm, roundedcorners=10, linewidth=0mm, titlewidthscale=0.7,%
%  bodywidthscale=0.9, bodyverticalshift=2cm, titleright]
% {Block outside of Columns}{Along with several options enabled}

\end{document}



\endinput
%%
%% End of file `tikzposter-template.tex'.
